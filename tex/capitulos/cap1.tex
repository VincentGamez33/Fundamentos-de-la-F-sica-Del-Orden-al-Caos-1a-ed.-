\chapter[Elasticidad]{ELASTICIDAD}
\startcontents
\printchaptertableofcontents

En las páginas que siguen, exploraremos un concepto central de la mecánica de sólidos: la \textbf{elasticidad}. Esta propiedad de los materiales es la base de numerosos campos de la ciencia y la ingeniería, permitiéndonos comprender y predecir cómo se comportan las estructuras bajo la acción de fuerzas externas. En este capítulo aprenderemos que es un cuerpo elástico, sus propiedades fundamentales y como se aplican los esfuerzos y deformaciones en situaciones de la vida cotidiana.

\section{Esfuerzos longitudinales}

La \textbf{elasticidad} es la parte de la estática\footnote{La estática es la rama de la mecánica que estudia los cuerpos en equilibrio.} que estudia cómo se deforman los cuerpos \textbf{elásticos}\footnote{La definición se muestra en la página \ref{}.} debido a la
aplicación de fuerzas externas que se ejercen sobre ellos.

\begin{definition}{}{}
    Entendemos por \textbf{deformación} el cambio en las dimensiones y/o la forma de un cuerpo causado por la aplicación de fuerzas externas ejercidas sobre él.
\end{definition}

Un \textbf{cuerpo rígido ideal} sería aquel que no sufre deformación alguna al ejercer fuerzas sobre él.

Consideremos un cuerpo de material homogéneo\footnote{Un material homogéneo es aquel que tiene la misma densidad en todas partes.}, sobre el cual se han aplicado fuerzas externas que lo deformaron.

Si al suprimir las fuerzas deformadoras sobre él, éste recupera del todo su forma y dimensiones originales decimos que el material es \textbf{perfectamente elástico}. Si las recupera en un alto porcentaje diremos que el material es \textbf{elástico}. Si no las recupera en absoluto diremos que el material es \textbf{perfectamente plástico} y si las recupera en un pequeño porcentaje diremos que el material es \textbf{plástico} y que sufrió una \textbf{deformación plástica}.

En el mundo real todos los materiales presentan comportamientos elásticos y también plásticos dependiendo de la magnitud de las fuerzas deformadoras. Si la magnitud de las fuerzas deformadoras es muy pequeña es casi seguro que el comportamiento de la mayoría de los materiales es elástico; pero si las magnitudes de esas fuerzas son grandes entonces tendremos deformaciones plásticas hasta llegar a la ruptura del material. Los materiales elásticos almacenan energía mecánica al ser deformados, mientras que los materiales plásticos convierten la energía mecánica de la deformación en energía interna al incrementar su temperatura. Una aplicación muy importante del estudio de las propiedades mecánicas de los materiales, como son las propiedades elásticas, es la deformación progresiva de los materiales que rodean el habitáculo de los pasajeros en un automóvil, con la finalidad de proteger la integridad física de los mismos.

Para explicar el fenómeno de la deformación de los cuerpos elásticos usamos una cantidad física escalar denominada \textbf{esfuerzo}.
\begin{definition}{}{}
    El esfuerzo es el cociente de la magnitud de la fuerza deformadora entre el área de alguna sección del cuerpo deformado. Es decir,
    $$\text{Esfuerzo}=\frac{\text{Fuerza}}{\text{Área}}.$$
    Además, la unidad física del esfuerzo en el Sistema Internacional de Unidades es el \textbf{Pascal}:
    $$1\text{ Pascal} = \frac{1\text{ Newton}}{m^2}$$
\end{definition}

Dependiendo de la dirección de las fuerzas deformadoras, los esfuerzos se podrán clasificar en \textbf{esfuerzos longitudinales} o \textbf{transversales}. Y en el caso en que la deformación sea causada por la presión ejercida por un fluido que rodea al cuerpo tendremos un \textbf{esfuerzo de presión hidrostática}.

\section{Esfuerzo de tensión}

Consideremos una barra recta de sección transversal rectangular uniforme\footnote{La sección transversal es la superficie que resulta de la intersección de un plano imaginario, perpendicular a la barra, con la barra misma. Cuando decimos que es uniforme significa que esta superficie es igual en toda la longitud de la barra.}, de longitud $L_0$, colocada sobre una mesa horizontal. A continuación, aplicamos un par de fuerzas externas $\F$, $-\F$ de igual magnitud, pero de sentido opuesto, en los extremos de la barra de manera que la estiran véase la Figura

De este modo, la barra de sección rectangular que reposa sobre la mesa se mantiene en equilibrio pues la suma de las fuerzas externas es igual a cero. Como resultado de la aplicación del par de fuerzas deformadoras en los extremos de la barra, tenemos un incremento $\DeltaL$ en su longitud original $L_0$.
\begin{definition}{}{}
    Llamamos \textbf{deformación} al incremento de la longitud, expresado por
    $$\DeltaL = L - L_0.$$
\end{definition}
\begin{definition}{}{}
    Denominamos \textbf{deformación unitaria} o \textbf{deformación relativa} al cociente
    $$D_U = \frac{\DeltaL}{L_0}.$$
\end{definition}

% \section{Esfuerzo de compresión}

% \section{Esfuerzo cortante o transversal}

% \section{Esfuerzo de presión hidrostática}

% \section{Relaciones esfuerzo-deformación}

% \subsection{Ley de Hooke}

% \subsection{Coeficiente de Poisson}

% \section{Relaciones entre coeficientes elásticos}

% \section{Torsión}

% \subsection{Péndulo de torsión}

% \subsection{Energía almacenada en una barra, eje o varilla torcida}

% \section{Flexión}

% \subsection{Flexión de una barra o una viga}

% \subsection{Flexión lateral de una barra}